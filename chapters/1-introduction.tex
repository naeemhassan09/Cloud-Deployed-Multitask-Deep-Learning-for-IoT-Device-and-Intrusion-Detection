% chapters/introduction.tex
\section{Background and Motivation}

\subsection{The IoT Landscape and Security Challenges}
By 2025, more than 75~billion Internet of Things (IoT) devices are expected to be active globally, spanning smart homes, healthcare systems, industrial automation, and critical infrastructure \parencite{Laghari2024}. This explosive growth represents one of the largest technological transitions in networked computing, shifting control from centralised servers to decentralised devices operating in dynamic, often unsecured environments.  
\vspace{0.2cm}

\begin{figure}[h!]
    \centering
\includegraphics[width=0.85\textwidth,keepaspectratio]{figures/Fig-1.png}
    \vspace{0.5cm}
    \caption{Global growth of Internet of Things (IoT) devices from 2015 to 2025. 
    Data adapted from \parencite{Statista2024}.}
    \label{fig:iot-growth}
\end{figure}

IoT ecosystems differ sharply from traditional enterprise networks: they integrate diverse devices with heterogeneous hardware, protocols, and firmware designs. Such heterogeneity limits standardisation and complicates unified security policies. The distributed nature of IoT infrastructure, often combined with limited update mechanisms and exposure to public networks, significantly expands the potential attack surface.
\vspace{0.2cm}
Cyberattacks against IoT deployments have increased in both frequency and sophistication. Common vectors include Distributed Denial of Service (DDoS), botnet infiltration, spoofing, ransomware, and data exfiltration attacks exploiting weak authentication or firmware vulnerabilities \parencite{Nazir2025, Patil2025}. Historical events such as the 2016 Mirai botnet attack, which compromised more than 600,000 consumer devices, revealed the cascading risks posed by poorly secured IoT nodes. Recent research indicates that attackers are adopting zero-day exploits and adaptive malware that leverage device-specific behaviours for persistence within IoT networks.
\vspace{0.2cm}
The consequences of IoT security breaches extend beyond data loss or system downtime. Compromised industrial controllers or medical sensors may directly endanger physical safety. Regulatory non-compliance under frameworks such as GDPR or CCPA can incur severe financial penalties, while reputational damage may lead to lasting business impact. The convergence of operational technology (OT) and information technology (IT) further amplifies the risk, as insecure IoT endpoints often act as entry points into broader enterprise networks.

\vspace{0.2cm}
\begin{table}[h!]
\centering
\begin{tabular}{@{}p{5cm}p{10cm}@{}}
\toprule
\textbf{Attack Type} & \textbf{Typical Example / Target} \\ \midrule
DDoS / Botnet & Mirai, BASHLITE—massive traffic floods via cameras or routers \\
Spoofing & ARP, DNS, or IP spoofing to hijack communication \\
Reconnaissance & Port scanning and OS fingerprinting to identify vulnerable devices \\
Web-Based & SQL injection, XSS in IoT control panels \\
Firmware Manipulation & Uploading malicious firmware via unsecured update channels \\ \bottomrule
\end{tabular}
\caption{Representative IoT attack categories \parencite{Gueriani2024}.}
\end{table}

\subsection{Limits of Traditional Intrusion Detection}
Signature-based intrusion detection remains the dominant security mechanism in conventional IT environments. These systems identify attacks by matching observed traffic patterns to known signatures. However, they fail when confronted with novel or polymorphic malware. Updating large rule sets in real time is computationally prohibitive for most IoT gateways or edge nodes. The diversity of communication protocols—ranging from MQTT and CoAP to Zigbee, LoRaWAN, and BLE—renders universal rule coverage unrealistic.  
\vspace{0.2cm}

Rule-based anomaly detection suffers similar rigidity. Although such systems encode expert heuristics, they lack adaptability and require manual tuning whenever new devices or behaviours emerge. Given the dynamic nature of IoT deployments, these static systems rapidly lose relevance.

\subsection{Machine Learning and the Single-Task Constraint}
Machine-learning (ML) methods improve over rule-based systems by inferring complex decision boundaries from data. Techniques such as Random Forests, Support Vector Machines, and early feedforward neural networks have demonstrated strong performance in distinguishing benign and malicious traffic. Nevertheless, their designs typically focus on a single analytic goal—either intrusion detection or device classification.  

This single-task paradigm presents several limitations:
\begin{enumerate}
    \item It ignores cross-task relationships; information about device identity can aid detection accuracy, while abnormal activity can flag compromised devices.
    \item Single-task models require complete retraining when the threat landscape changes, introducing downtime and resource overhead.
    \item Maintaining multiple models in production increases latency, memory usage, and orchestration complexity, which is impractical for resource-constrained IoT gateways.
\end{enumerate}

\subsection{The Case for Multitask Deep Learning}
Deep learning has advanced intrusion detection by automating feature extraction from raw network traffic. Hybrid models that combine Convolutional Neural Networks (CNNs) and recurrent or attention-based layers capture both spatial correlations and temporal dynamics. For instance, CNN–LSTM architectures have achieved detection accuracies above 98\% on benchmark datasets such as CICIoT 2023 by modelling spatial–temporal dependencies \parencite{Gueriani2024}. Transformers further extend this capability through self-attention, learning contextual dependencies across long traffic sequences \parencite{Ullah2023}.  
\vspace{0.2cm}

Multitask learning (MTL) generalises these ideas by sharing feature representations across related tasks. A single CNN–Transformer encoder can feed two lightweight heads: one for device identification and one for intrusion detection. MTL offers three major advantages:
\begin{itemize}
    \item \textbf{Efficiency:} Shared layers reduce redundant computation, improving inference speed and memory efficiency.
    \item \textbf{Generalisation:} Shared inductive bias helps the model adapt to unseen devices or attack types.
    \item \textbf{Operational simplicity:} A unified model simplifies deployment pipelines and monitoring.
\end{itemize}

\textbf{[Insert Figure 2: Conceptual diagram of the proposed multitask CNN–Transformer architecture.]}  
\emph{Suggestion: show shared encoder block feeding two heads—(1) IoT device ID, (2) intrusion classification—annotated with flow of data from input features to predictions.}

\subsection{Contribution and Integration into Cloud Environments}
Building on these insights, this research develops a \emph{cloud-deployed multitask CNN–Transformer} framework capable of simultaneous IoT device identification and intrusion detection. The model is trained and evaluated on the CIC IoT-IDAD 2024 dataset, which includes benign and attack flows across multiple IoT device classes and attack families.  
\vspace{0.2cm}

Key contributions include:
\begin{itemize}
    \item Implementing a multitask architecture optimised for accuracy and latency under realistic network workloads.
    \item Benchmarking against strong single-task baselines—XGBoost, BiLSTM, and TabNet—to quantify trade-offs in computational efficiency and scalability.
    \item Deploying the trained model as a containerised FastAPI microservice on AWS Elastic Container Service (ECS) with automated CI/CD pipelines for reproducibility.
    \item Integrating Prometheus and Grafana for continuous observability and performance tracking in cloud production environments.
\end{itemize}

This approach bridges the gap between experimental ML research and production-grade IoT security systems.

\section{Problem Statement}
IoT infrastructures present distinct challenges relative to traditional networks. They consist of thousands of low-power devices—sensors, cameras, thermostats, or controllers—communicating asynchronously through heterogeneous protocols. Variations in manufacturer security practices, weak firmware validation, and limited encryption support exacerbate vulnerabilities. Firmware updates are inconsistent, leaving many devices susceptible to known exploits long after fixes exist.

Conventional IDS frameworks, designed for well-provisioned enterprise servers, do not scale down to these environments. Their assumptions of stable topology and uniform configuration fail in distributed IoT systems where devices may appear, disappear, or behave differently based on context. Attack vectors such as botnet recruitment, firmware manipulation, and side-channel exploitation require models that generalise beyond static signatures.

Furthermore, treating device identification and intrusion detection as separate functions doubles processing overhead. Independent models overlook the interdependence between device behaviour and threat manifestation. This redundancy increases latency and weakens detection sensitivity.

Most research still evaluates models offline on benchmark datasets, ignoring deployment realities. Few studies report inference latency, resource usage, or continuous monitoring metrics. As a result, promising prototypes remain laboratory artefacts, not deployable security solutions.

Hence, there is a critical need for a unified, scalable, and cloud-operational deep-learning framework that can perform both tasks concurrently, maintaining low latency, interpretability of performance metrics, and operational reliability.

\section{Research Aim}
The aim of this research is to design, implement, and evaluate a multitask CNN–Transformer framework for concurrent IoT device identification and intrusion detection. The framework should achieve high classification accuracy and low inference latency while supporting scalable deployment through FastAPI microservices on AWS ECS, automated CI/CD, and real-time monitoring with Prometheus and Grafana.

\section{Research Objectives}
\begin{enumerate}
    \item \textbf{Dataset Analysis and Preparation:} Conduct exploratory data analysis on the CIC IoT-IDAD 2024 dataset to identify device categories, attack types, feature distributions, and class imbalance. Implement preprocessing pipelines including normalisation, feature encoding, and stratified splitting.
    \item \textbf{Multitask Architecture Development:} Design a hybrid CNN–Transformer encoder with task-specific heads for device identification and intrusion detection. Experiment with depth, attention mechanisms, and regularisation strategies.
    \item \textbf{Model Training and Optimisation:} Configure multitask loss functions and optimisation algorithms. Apply cross-validation and evaluate model convergence, preventing overfitting through dropout and early stopping.
    \item \textbf{Baseline Benchmarking:} Train and compare XGBoost, BiLSTM, and TabNet models to the proposed framework using accuracy, F1-score, ROC-AUC, latency, and memory metrics.
    \item \textbf{Production Deployment:} Develop a FastAPI service exposing REST endpoints. Containerise the model with Docker and deploy to AWS ECS with load balancing and health checks. Automate testing and deployment through GitHub Actions.
    \item \textbf{Monitoring and Observability:} Integrate Prometheus and Grafana dashboards to monitor request throughput, latency, system load, and error rates. Configure alerting for anomaly detection.
    \item \textbf{Performance Evaluation:} Conduct stress tests under varying network loads to assess scalability and resilience. Analyse outcomes to identify limitations and areas for future research.
\end{enumerate}

\section{Scope and Delimitations}
\subsection{Dataset}
\textbf{Included:} CIC IoT-IDAD 2024, a publicly available dataset covering multiple IoT devices and attack categories.  
\textbf{Excluded:} Proprietary or undisclosed datasets, synthetic data generation, or other domain-specific traffic not aligned with the target scenario.

\subsection{Model Architecture}
\textbf{Included:} Multitask CNN–Transformer with shared encoder; baselines XGBoost, BiLSTM, TabNet.  
\textbf{Excluded:} Generative adversarial networks, reinforcement learning, LLM-based log analysis, graph neural networks, and federated learning due to time and computational constraints.

\subsection{Deployment}
\textbf{Included:} Docker containerisation, AWS ECS orchestration, CI/CD (GitHub Actions), FastAPI interface, and monitoring with Prometheus and Grafana.  
\textbf{Excluded:} Edge deployment, FPGA/ASIC acceleration, or non-AWS platforms (Azure, GCP).

\subsection{Evaluation Metrics}
\textbf{Included:} Accuracy, precision, recall, F1-score, ROC-AUC, latency (mean/percentile), throughput, memory utilisation, and confusion matrices.  
\textbf{Excluded:} Energy consumption, cost analysis, adversarial robustness, and formal verification.

\subsection{Regulatory and Compliance}
\textbf{Included:} Discussion of privacy (GDPR, CCPA) and secure handling of traffic data.  
\textbf{Excluded:} Full compliance audits or certification studies (ISO 27001, SOC 2).

This scoped design ensures depth and feasibility within academic and resource constraints while producing a reproducible, cloud-operational multitask intrusion detection framework.

% End of introduction.tex